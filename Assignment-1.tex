\documentclass{article}
\usepackage[utf8]{inputenc}

\title{Assignment1}
\author{Mukul Kuamr Yadav}
\date{September 2020}

\renewcommand{\baselinestretch}{1.5}

\begin{document} 

\maketitle

\noindent \textbf {Question No. 62:} 
 A line perpendicular to the line segment joining the points (1,0) and (2,3) divides it into the ratio 1:n. Find the equation of the line. 
\par
\setlength{\parskip}{1em}
\noindent \textbf {Solution:} Let Coordinate of Point A = (1,0) and B = (2,3). Assume C will be the point there a line intersect the line segment AB. The coordinate of Point C will be 
$
\frac{n+2}{n+1}, \frac{3}{n+1}
$
\par
Slope of line AB  = $\frac{3-0}{2-1}$ = 3
\par
let slope of the line m. If two lines are perpendicular to each other their multiplication of slope  = -1. Therefore, m = $\frac{-1}{3}$ 
\par
equation of new line would be y=mx+c.i.e. $ y= \frac{-1}{3}x+c $. The new line will pass through the point C. so, $ c= \frac{3}{n+1} + \frac{1}{3}(\frac{n+2}{n+1})

\par
Therefore, the final equation of line will be $ y = (\frac{-1}{3})x + \frac{3}{n+1} + \frac{1}{3}(\frac{n+2}{n+1}).
\par
NOTE: This is my first LaTex code. 
\end{document}
