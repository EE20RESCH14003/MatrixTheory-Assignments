\documentclass[a4paper,12pt]{article}
\usepackage{amsmath}
\usepackage{tikz}

\title{\textbf{Assignment-1} \vspace{5mm}\\ Mukul Kumar Yadav}  

\begin{document}
\maketitle
\begin{flushleft}	

\textbf{Problem Statement (Q.No.62):} \vspace{5mm}\\
\textit{A line perpendicular to the line segment joining the points (1,0) and (2,3) divides it into the ratio 1:n. Find the equation of the line.} \vspace{5mm} 

\textbf{Solution:} \vspace{5mm}\\
\begin{center}
    
\begin{tikzpicture}

\coordinate [label=right:$A$] (A) at (1,0);
\coordinate [label=right:$B$] (B) at (2,3);
\coordinate [label=right:$P$] (P) at (1.5,1.5);
\coordinate [label=right:$R$] (R) at (-2,4);
\coordinate [label=left:$O$] (O) at (0,0);
\draw (A) -- (B);
\draw (P) -- (R);
\draw [dash dot] (O) - -(A)
\draw [dash dot] (O) - -(P)
\draw [dash dot] (O) - -(B)
\end{tikzpicture} 
\end{center} \\
Given that
$A=
\begin{pmatrix} 
1 \\
0
\end{pmatrix} and B=
\begin{pmatrix} 
2 \\
3
\end{pmatrix}.

Let P= \begin{pmatrix} 
x \\
y
\end{pmatrix} and origin O = \begin{pmatrix} 
0 \\
0
\end{pmatrix}. Since the line RP intersect the line AB in 1:n ration, then 
$ \frac{AP}{PB} = \frac{1}{n} $. PB = n.AP or OB-OP = n(OP-OA). Solving this vector equation, OP = \dfrac{OB+n.OA}{n+1}\\

Therefore, \begin{pmatrix} 
x \\
y
\end{pmatrix} = \begin{pmatrix}
\dfrac{(n+2)}{(n+1)} \\ 
\dfrac{3}{n+1} 
\end{pmatrix}\\
Vector equation of Line RP, $\textbf{r} = \textbf{p} + \lambda. \textbf{d} $, where  \textbf{p} is point on the line, \lambda \ is \ constant, and \ \textbf{d} \  is \ direction \ vector \ of \ the \ line. \\

Since the line segment AB and the line RP is perpendicular to each other then dot product of both the vectors will be zero.\\

AB.d = 0. \Longrightarrow 
$\begin{pmatrix} 
2-1 \\
3-0
\end{pmatrix}^T. \begin{pmatrix} 
x \\
y
\end{pmatrix} = 0$
$\Longrightarrow x+3y = 0. $
x= 3 and y = -1. Therefore, direction vector \textbf{d} = \begin{pmatrix} 
3 \\
-1
\end{pmatrix} $ \\ 
Final vector equation of the line is \textbf{r} = 
\begin{pmatrix}
\dfrac{(n+2)}{(n+1)} \\ 
\dfrac{3}{n+1} 
\end{pmatrix} + \lambda. \begin{pmatrix} 
3 \\
-1
\end{pmatrix}

\end{flushleft}

\end{document}